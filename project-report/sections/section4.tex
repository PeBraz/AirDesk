\section{Implementação}
O \textit{churn} pode provocar anomalias na comunicação entre utilizadores. Uma das limitações é ser preciso estar na actividade \textit{MyWorkspacesActivity.java} para a ligação entre utilizadores começar. Se por algum motivo externo(e.g falha na rede), a ligação entre utilizadores falhar é preciso voltar à activiade anteriormente mencionada para começar a ligação de novo. Outro dos problemas associados à implementação em dispositivos reais é a dificuldade em testar o projecto em dispositivos com diferentes versões do Android. Relativamente ao \textit{Wi-Fi} o ideal era que a ligação fosse transparente. Contudo, não é possível detectar quais os dispositivos que estão a correr a nossa aplicação. Daí, a ligação ter de ser feita manualmente. Outra limitação do nosso projecto é que na parte distribuída não existe qualquer distinção entre \textit{workspaces} públicos e privados.

A plataforma Android tem disponivel diversos mecanismos de armazenamento persistente. Optámos por guardar a maioria dos dados numa base de dados \textit{SQLLite}. A base de dados permite-nos procurar registos eficientemente e alterá-los com facilidade. Além disso permite fazer \textit{queries} por um campo específico da tabela.

Para manter o estado do último utilizador na aplicação, utilizámos \textit{shared preferences} pois para esta operação um mecanismo \textit{key -> value store} é o suficiente.

Os ficheiros são guardados no sistema de ficheiros do telemóvel(\textit{internal storage}). 


